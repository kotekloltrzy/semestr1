\documentclass[a4paper,15pt]{book}
\usepackage{latexsym}
\usepackage[MeX]{polski}
\usepackage[utf8]{inputenc}
\author{M.~Broda}
\title{Księga Magii~i~Iluzji Tom~I}
\frenchspacing
\begin{document}
\pagestyle{plain}
\maketitle
\thispagestyle{empty}
\tableofcontents
\part{Pierwsze zajęcia}
\begin{center}Praca Domowa N1 \\

Michał Broda \\

\end{center}
\scshape
\small 
1.~Przykład~~tekstu~po-angielsku:~Aesop~"The~Hare~and~The~Tortoise"

\upshape

One day the Hare laughed at the short feet and slow speed of the Tortoise. The

Tortoise replied:

\textit{"You may be as fast as the wind, but I will beat you in a race"}

The Hare thought this idea was impossible and he agreed to the proporsal. It

was agreed that the Fox should choose the course and decide the end.

The day for the race came, and the Tortoise and Hare started together.

The Tortoise never stopped for a moemnt, walking slowly but steadily,

right to the end of the course. The Hare ran fast and stopped to lie down for a rest.

But he fell asleep. Eventually, he woke up and ran as fast as he could. But

when he reached the end, he saw Tortoise there already, sleeping comfortably

after her effort.

\begin{center}
2. Rozmiary Czcionki \\

\end{center}
My przeczytaliśmy wszystkie polecenia na stronie: \\

https://www.overleaf.com/learn/latex/Font\_sizes\%2C\_families\%2C\_and\_styles \\

Teraz możemy to wykorzystać. \\

Czcionka jest tutaj bardzo mała, trzeba powiększyć ten tekst. Teraz już lepiej, ale jeszcze nie \\

zwykle. O, tutaj jest normalny tekst! A tutaj jest większa czcionka dla czegoś \\

specjalnego. I, oczywiście, teraz spróbujemy my \\

największą czcionką. \\

\newpage
\part{Drugie zajęcia}
\begin{center}
WPROWADZENIE W TRYB MATEMATYCZNY \\

Michał Broda \\
1. SYMBOLY MATEMATYCZNY \\

\end{center}
1.1 Sumy, iloczyny i całki. \\


1.2. Funkcja Eulera. \\

\end{document}

